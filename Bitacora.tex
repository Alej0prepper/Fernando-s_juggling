\documentclass{article}
\usepackage[margin=1in]{geometry} % Márgenes de 1 pulgada (puedes ajustarlos según tus necesidades)
\usepackage{amsmath}
\usepackage{amssymb}
\usepackage{graphicx}

\begin{document}

\section*{Bitácora del Proyecto}

\subsection*{Semana 2: Orientación del Trabajo}

Iniciamos la asignatura con una sesión de orientación. ¡Qué interesante! Un trabajo sobre malabares, aunque mencionaron algo de grafos, espero que no sea muy complicado. 🤹‍♂️

\subsection*{Semana 3: Primeras Ideas}

Adrián habló con Fernando, y ya tenemos una idea más clara de lo que debemos hacer. Decidimos que vamos a desarrollar un MVP (Producto Mínimo Viable) para presentárselo al profesor. 💡

\subsection*{Semana 5: Descubrimientos Iniciales}

Descubrimos que los malabares se pueden representar con números. ¡Waoooo! Pero... ¿qué significan estos números? Lo que sabemos es que se llaman siteswaps. 🔢

\subsection*{Semana 6: Avances en el Concepto}

Creemos que, con los mismos números de los malabares, podemos quitar una pelota del siteswap y seguir representándolo si el resultado es válido. 🤔

\subsection*{Semana 7: Primer MVP}

¡Yupii! Tenemos un MVP con visual y todo. A veces las pelotas se mueven de forma errática, pero todo bien. Vamos a reunirnos con Fernando para mostrarle nuestros avances. 🎉

\subsection*{Semana 8: Revisión y Corrección de Rumbo}

😞 Vimos a Fernando y nos regañó por no habernos comunicado con él antes. Resulta que lo que hemos hecho no era lo que él quería. Él quería trabajo con grafos y nosotros nos habíamos inventado nuestra forma de enseñar los siteswaps. Pero bueno, la buena noticia es que a partir de ahora sabemos que vamos por buen camino. 💪

\subsection*{Semana 10: Progreso con el Subgrafo}

Ya estamos generando el subgrafo con una pelota de menos. Me parece que la idea está bastante buena. Se lo comunicamos al profesor y, al parecer, le gustó. Tenemos un par de dudas que vamos a aclarar en cuanto nos veamos. 👍

Esperamos seguir mejorando y ajustando nuestro proyecto para cumplir con todas las expectativas y requerimientos. 🚀

\end{document}
